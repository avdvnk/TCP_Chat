\documentclass{article}
\usepackage[14pt]{extsizes} 
\usepackage[utf8]{inputenc}
\usepackage[russian]{babel}
\usepackage{graphicx}
\graphicspath{{pictures/}}
\DeclareGraphicsExtensions{.pdf,.png,.jpg}
\usepackage{xcolor}
\usepackage{hyperref}
\usepackage{listings}
\usepackage{float}
\usepackage{indentfirst}
\usepackage[left=2.5cm, right=1.5cm, vmargin=2.5cm]{geometry}

\begin{document}

  \begin{center}
	   ГУАП\\
	   КАФЕДРА № 51
  \end{center}
  \begin{center}
\begin{tabular}{clcll}
& & & & \\
\multicolumn{1}{l}{ПРЕПОДАВАТЕЛЬ} & \multicolumn{1}{c}{} & & & \\ \cline{1-3}
\multicolumn{1}{|c|}{доцент, к.т.н.} & \multicolumn{1}{l|}{} & \multicolumn{1}{c|}{Линский Е.М.} & & \\ \cline{1-3}
\multicolumn{1}{|c|}{\begin{tabular}[c]{@{}c@{}}должность , уч. степень,\\ звание\end{tabular}} & \multicolumn{1}{c|}{подпись, дата} & \multicolumn{1}{c|}{инициалы, фамилия} &  &  \\ \cline{1-3}
\end{tabular}
\end{center}

\vspace{5cm}

  \begin{center}
	ОТЧЕТ О ЛАБОРАТОРНОЙ РАБОТЕ № 12\\
	СОЗДАНИЕ ПРОГРАММЫ НА ЯЗЫКЕ JAVA\\
	\vspace{1cm}
	по курсу: ТЕХНОЛОГИИ ПРОГРАММИРОВАНИЯ\\
  \end{center}
	
	\vspace{4cm}
	\begin{center}
\begin{tabular}{cccll}
& \multicolumn{1}{l}{} & & &  \\
\multicolumn{1}{l}{РАБОТУ ВЫПОЛНИЛ} & & & &  \\ \cline{1-4}
\multicolumn{1}{|c|}{СТУДЕНТ ГР. №} & \multicolumn{1}{c|}{5511} & \multicolumn{1}{c|}{} & \multicolumn{1}{c|}{Вдовенко А.} & \\ \cline{1-4}
\multicolumn{1}{|c|}{} & \multicolumn{1}{c|}{} & \multicolumn{1}{c|}{подпись, дата} & \multicolumn{1}{c|}{\begin{tabular}[c]{@{}c@{}}инициалы,\\ фамилия\end{tabular}} &  \\ \cline{1-4}
\end{tabular}
\end{center}
	\vspace{1cm}
  \begin{center}
	Санкт-Петербург 2017
  \end{center}
\thispagestyle{empty}
\newpage
	\textbf{Задание:}
	\\Написать текстовый многопользовательский чат.
	\begin{enumerate}
		\item Пользователь управляет клиентом. На сервере пользователя нет. Сервер занимается пересылкой сообщений между клиентами.
		\item По умолчанию сообщение посылается всем участникам чата.
		\item Есть команда послать сообщение конкретному пользователю (@senduser Vasya).
		\item Есть команда @info которая присылает клиенту список всех пользователей, находящихся онлайн.
	\end{enumerate}
	Программа работает по протоколу TCP.\\

	
	\textbf{Дополнительное задание:}
	\\Написать бота, который переодически посылает сообщения всем пользователям с текущем временем, а если написать боту, то отвечает эхом (копией исходящего сообщения), если в общем чате кто-то из клиентов написал <<Hello>>, то бот отвечает своей коронной фразой.\\
	
	
	\textbf{Инструкция:}
	\\Запуск программы происходит в следующем порядке:
		\begin{itemize}
			\item {Запуск сервера, выбор порта для сервера.}
			\item {Запуск клиентов, ввод адреса сервера, ввод порта, ввод имени пользователя.}
			\item {Ведение диалога. Использование команд выхода, отправление приватного сообщения, информации о присутствующих пользователях.}
		\end{itemize}
	
	
	\textbf{Тестирование:}
	\begin{enumerate}
		\item Используем команду @quit, программа клиента завершается
		\item Используем команду @sendUser\\
		С клиента Katya пошлём сообщение клиенту Vasya, для этого вводим: <<@senduser Vasya: Hello!>>.\\
		В результате этой команды только у клиента с именем Vasya появится сообщение: <<[Private] Katya: Hello!>>.
		\item Проверяем работу бота. Пишем в общий чат сообщение <<Hello!>>\\
		Бот отвечает в общий чат фразой <<O-ho-ho, hello, my friends!>>\\
		Вводим в консоль: @sendUser Bot: <<Привет, бот! Как дела?>>\\
		В ответ приходит сообщение от бота: <<[Private] Bot: Привет, бот! Как дела?>>
	\end{enumerate}
\end{document}
